\documentclass{bioinfo}
\copyrightyear{2016}
\pubyear{2016}

\usepackage{hyperref}

\begin{document}
\firstpage{1}

\title[short Title]{Integrated Prediction of Protein Function and Interactions}
\author[Sample \textit{et~al}]{Claudio Sacc\`a\,$^{1}$, Stefano Teso\,$^{2}$, Michelangelo Diligenti\,$^{1*}$, and Andrea Passerini\,$^2$\footnote{to whom correspondence should be addressed}}
\address{
    $^{1}$Dipartimento di Ingegneria dell'Informazione e Scienze Matematiche, University of Siena, Siena, Italy\\
    $^{2}$Dipartimento di Ingegneria e Scienza dell'Informazione, University of Trento, Trento, Italy
}
\history{Received on XXXXX; revised on XXXXX; accepted on XXXXX}
\editor{Associate Editor: XXXXXXX}

\maketitle

\begin{abstract}
\section{Motivation:}
\section{Results:}
\section{Availability:}
\section{Contact:}
\end{abstract}



\section{Introduction}

\begin{itemize}

    \item Protein function and GO hierarchy

    \item Protein--protein interactions

    \item Importance of genome-wide prediction

    \item Relationships between function and interactions

    \item Statistical prediction methods for relational data with multiple
    features

    \item Structure of the genome-wide predictor (intro to SBR)

    \item Performance highlights

    \item Structure of the paper

\end{itemize}



\begin{methods}

\section{Methods}

\subsection{Prediction Model}

WRITEME

\end{methods}



\section{Results}

\subsection{Dataset construction}

\begin{itemize}

    \item Dataset taken from manually curated portions of the SGD database. We
    consider all validated ORFs at least 50 residues long.

    \item Sequences are filtered with CD-HIT with a threshold of 80.

    \item GO annotations: taken from BioGRID, IEA annotations removed.

    \item Interactions: taken from BioGRID via SGD. Extracted from the
    literature.

    \item We compute the normalized average of:

    \begin{itemize}

        \item Genetic context (colocalization) kernel

        \item Coexpression kernel

        \item Complex kernel

        \item Domain kernel

        \item Complex kernel

        \item Profile kernel

    \end{itemize}

    \item We compute the pairwise (tensor product) kernel

\end{itemize}

\subsection{Evaluation setting}

\begin{itemize}

    \item Negative functions XXX

    \item Negative interactions sampled from the complement of the known
    positive interactions, minus the STRING database to avoid possible
    positives.

    \item PF and PPI balanced fold construction (each test protein has
    at least one unknown interaction that we aim at predicting)

\end{itemize}

\paragraph{Protein function prediction}

\begin{itemize}

    \item Competitors: BLAST sequence transfer and nearest neighbor, as in
    CAFA2; plus GoFDR and possibly other high-rank CAFA2 methods

    \item Performance metrics: as in CAFA2, protein-centric (Fmax and Smin)
    and term-centric (AUC).

\end{itemize}

\paragraph{Protein--protein interaction prediction}

\begin{itemize}

    \item Competitors: ?

    \item Performance metrics: ?

\end{itemize}



\section{Discussion}

Make sure to mention that our method is tailored towards genome-wide
prediction, so CAFA2 is out of scope.

WRITEME



\section{Conclusion}

WRITEME



\section*{Acknowledgements}

WRITEME



\paragraph{Funding\textcolon}

WRITEME



\bibliographystyle{unsrt} % FIXME use bioinformatics.bst
\bibliography{document}

\end{document}
