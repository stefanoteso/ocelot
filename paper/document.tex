\documentclass{bioinfo}
\copyrightyear{2016}
\pubyear{2016}

\usepackage{hyperref}

\begin{document}
\firstpage{1}

\title[short Title]{Integrated Prediction of Protein Function and Interactions}
\author[Sample \textit{et~al}]{Claudio Sacc\`a\,$^{1}$, Stefano Teso\,$^{2}$, Michelangelo Diligenti\,$^{1*}$, and Andrea Passerini\,$^2$\footnote{to whom correspondence should be addressed}}
\address{
    $^{1}$Dipartimento di Ingegneria dell'Informazione e Scienze Matematiche, University of Siena, Siena, Italy\\
    $^{2}$Dipartimento di Ingegneria e Scienza dell'Informazione, University of Trento, Trento, Italy
}
\history{Received on XXXXX; revised on XXXXX; accepted on XXXXX}
\editor{Associate Editor: XXXXXXX}

\maketitle

\begin{abstract}

\section{Motivation:}
We want to win obvious improvement of prediction accuracy so we can publish
and get money!

\section{Results:}
Not my problem!

\section{Availability:}
Elsewhere!

\section{Contact:} \href{Andrea Passerini}{passerini@disi.unitn.it}
\end{abstract}



\section{Introduction}

Fix this dot:
$$ \odot $$



\section{Approach}

Fool you into fixing the dot, then run away with the money/girls/fast car.

\section{Background}

\paragraph{Function}

\paragraph{Interactions}

\paragraph{Hybrid competitors?}

\begin{methods}



\section{Methods}

\subsection{Dataset}

\begin{itemize}
    \item Dataset taken from manually curated portions of the SGD database. We
    consider all validated ORFs at least 50 residues long.

    \item Sequences are filtered with CD-HIT with a threshold of 80.

    \item GO annotations: taken from BioGRID, we keep manual annotations only.

    \item Interactions: taken from BioGRID via SGD. Extracted from the
    literature.

    \item Negative interactions sampled from the complement of the known
    positive interactions, minus the STRING database to avoid possible
    positives.

\end{itemize}



\section{Experimental analysis}

\begin{itemize}

    \item Two comparisons with GOstrut:

    \begin{itemize}

        \item Compare GOstrut and SBR with our kernel

        \item Compare GOstrut and SBR with their kernel

        \item Potentially figure out the best combination of kernels

    \end{itemize}

    \item Genome-wide experiment

\end{itemize}



\section{Empirical Analysis}

\subsection{Versus GOstrut: our kernels}

\begin{itemize}

    \item Genetic context (colocalization) kernel

    \item Coexpression kernel

    \item Complex kernel

    \item Domain kernel

    \item Complex kernel

    \item Profile kernel

    \item Average kernel

\end{methods}

\subsection{Versus GOstrut: their kernels}

\begin{itemize}

    \item All-vs-all blast linear kernel

    \item Localization kernel with Wolf PSort

    \item Transmembrane helices linear kernel

    \item N- and C- termini composition kernel

    \item Low-complexity segment kernel

    \item PPI kernel XXX

    \item Gene expression kernel XXX

    \item Co-mention kernel XXX

\end{methods}

\subsection{Genome-wide experiment on yeast}

\begin{itemize}

    \item Huge number of possible interactions; we can not possibly hope
    to compute the full pairwise Gram matrix

    \item As long as we do not have rules between interactions, we only need
    the train-train and train-test portion of the Gram matrix.

\end{itemize}


\section{Discussion}

WRITEME



\section{Conclusion}

WRITEME



\section*{Acknowledgements}

WRITEME



\paragraph{Funding\textcolon}

WRITEME



\bibliographystyle{unsrt} % FIXME use bioinformatics.bst
\bibliography{document}

\end{document}
