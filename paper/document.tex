\documentclass{bioinfo}
\copyrightyear{2016}
\pubyear{2016}

\usepackage{hyperref}

\usepackage[usenames,dvipsnames]{xcolor}
\definecolor{azure}{rgb}{0.0, 0.5, 1.0}
\newcommand{\andrea}[1]{{\bf \textcolor{azure}{Andrea: #1}}}
\definecolor{alizarin}{rgb}{0.82, 0.1, 0.26}
\newcommand{\stefano}[1]{{\bf \textcolor{alizarin}{{Stefano: #1}}}}
\definecolor{antiquefuchsia}{rgb}{0.57, 0.36, 0.51}
\newcommand{\luca}[1]{{\bf \textcolor{antiquefuchsia}{Luca: #1}}}
\definecolor{applegreen}{rgb}{0.55, 0.71, 0.0}
\newcommand{\michleangelo}[1]{{\bf \textcolor{applegreen}{Michelangelo: #1}}}
\definecolor{munsell}{rgb}{0.0, 0.5, 0.69}
\newcommand{\claudio}[1]{{\bf \textcolor{munsell}{{Claudio: #1}}}}



\begin{document}
\firstpage{1}

\title[short Title]{Integrated Prediction of Protein Function and Interactions}
\author[Sacc\`a \textit{et~al}]{Claudio Sacc\`a\,$^{1}$, Stefano Teso\,$^{2}$, Luca Masera\,$^{2}$, Michelangelo Diligenti\,$^{1*}$, and Andrea Passerini\,$^2$\footnote{to whom correspondence should be addressed}}
\address{
    $^{1}$Dipartimento di Ingegneria dell'Informazione e Scienze Matematiche, University of Siena, Siena, Italy\\
    $^{2}$Dipartimento di Ingegneria e Scienza dell'Informazione, University of Trento, Trento, Italy
}
\history{Received on XXXXX; revised on XXXXX; accepted on XXXXX}
\editor{Associate Editor: XXXXXXX}

\maketitle



\begin{abstract}
\section{Motivation:}
\section{Results:}
\section{Availability:}
\section{Contact:}
\end{abstract}



\section{Introduction}

\begin{itemize}

    \item Importance of genome-wide prediction of interactions and function

    \item Brief recap on protein function and GO hierarchy

    \item Brief recap on protein--protein interactions

    \item Relationships between function and interactions

    \item Statistical prediction methods for relational data with multiple
    features

    \item Structure of the genome-wide predictor (intro to SBR)

    \item Results highlights

    \item Structure of the paper

\end{itemize}



\begin{methods}

\section{Methods}

\subsection{Prediction Model}

\begin{itemize}

    \item High-level overview of joint function-interaction prediction at
    the genome scale

    \item We expand on the approach of Sacc\`a and colleagues~\cite{sacca2014ppisbr}

    \item Predicates: $\text{\texttt{bound}}(p,p'), \text{\texttt{func}}(p,t)$.

    \item Rules: (1) parents imply the disjunction of their childrent, (2)
    children imply all their parents, (3) interaction implies the same function
    within the same level.

    \item \stefano{bin terms are very unlikely in the final genome-wide
    experiment}

    \item Kernel-based predictions

    \item Constraints for function and interaction prediction

    \item Pairwise kernels

\end{itemize}

\subsection{Mathematical Formulation}

WRITEME

\end{methods}



\section{Results}

\subsection{Dataset construction}

In order to evaluate our proposed method, we built a comprehensive genome-wide
dataset. All data was retrieved in August 2014.  The source code of the data
processing pipeline, as well as the resulting dataset, are available at
\url{WRITEME}.

\paragraph{Annotations.} Protein sequences were taken from the Saccharomyces
Genome Database (SGD)~\cite{cherry2012sgd}. Only validated ORFs at least 50
residues long were retained. The sequences were redundancy reduced with
CD-HIT~\cite{fu2012cdhit}, using a threshold of 80, producing a set of 4865
proteins.

Functional annotations were also taken from SGD, while the GO hierarchy
definition was taken from the Gene Ontology Consortium website\footnote{
\url{http://geneontology.org/page/download-ontology}}. Following common
practice~\cite{gong2016gofdr} \stefano{more references}, we removed
automatically assigned (IEA) GO annotations. Obsolete GO annotations (resulting
from mismatches between SGD and the GO hierarchy) were removed as well. All
annotations were propagated to the root.

\stefano{Negative annotations were computed using the SNOB algorithm~\cite{youngs2014negative}.}

\stefano{add annotation statistics here}

The protein--protein interaction network (PIN) was taken from
BioGRID~\cite{chatr2015biogrid}. Only manually curated physical interactions
were kept. After symmetrizing the interactions, the resulting network accounted
for a total of 34611 known interacting protein pairs. Due to the general lack
of shared repositories of known non-interacting protein pairs in
yeast\footnote{Unfortunately, resources such as the
Negatome~\cite{blohm2013negatome} do not yet provide enough negative annotation
for our goals.}, we resorted to sampling candidate non-interactions from the
complement of the positive PIN.  This procedure is well-justified by the
overwhelming proportion of true non-interactions in the
complement~\cite{park2011revisiting}. The procedure was as follows. First, we
computed the complement of the positive PIN. To minimize the chance of sampling
false non-interactions, we subtracted from the complement all the known
(physical or functional) interactions annotated in either BioGRID or
STRING~\cite{franceschini2013string91,szklarczyk2014string10} (version 9.1).
Finally, we sampled 34611 non-interactions from the remaining candidate pairs
uniformly at random. The full set of interaction annotations is taken to be the
union of the above positive and negative sets.

\paragraph{Kernels.} In order to drive our learning procedure, we computed a
number of kernels between proteins and protein--protein pairs. We focused on
protein features that were shown to contribute useful information in several
studies, including gene co-localization and co-expression, protein complexes,
protein domains, and conserved residues. We next provide details of the
individual kernels.

Gene co-localization is known to influence the likelihood of proteins having an
interaction. The correlation is especially noticeable in prokaryote organisms,
but it is still significant in eukaryotes, such as yeast \stefano{add ref}.
This information is captured through the {\em co-localization} kernel, defined
as follows: the kernel $k_{ij}$ between proteins $i$ and $j$ is given by
%
$k_{ij} := \exp\left( - \gamma | x_i - x_j | \right)$,
%
where $x_i$ and $x_j$ are the absolute positions of the genes encoding the two
proteins. The kernel decreases as the distance between the genes increases.
Proteins encoded by different chromosomes have zero similarity.

Gene expression levels in cell-cycle and environmental response were also shown
to provide valuable information \stefano{add ref}. We computed a correlation
kernel over the expression levels measured in two different experiments
(\cite{spellman1998comprehensive} and \cite{gasch2000genomic}).

We computed a diffusion kernel~\cite{kondor2002diffusion} over
the yeast protein complex catalogue of~\cite{pu2009up}. It determines the
correlation between proteins through a diffusion process over the protein
complex network, setting the $\beta$ parameter to 1.

We computed a set/sparse kernel \stefano{only keep the better one} over
inferred protein domains, derived using InterPro
hits~\cite{mitchell2015interpro}.

We computed a profile kernel~\cite{kuang2005profile,hamp2013accelerating} over
all the proteins. The PSSMs were obtained with PSI-BLAST (default parameters,
two iterations) over NCBI's NR database. Please see \cite{kuang2005profile}
for additional details.

Each kernel defines a $4865 \times 4865$ protein--protein similarity (Gram)
matrix. The Gram matrices were first normalized by applying the
transformation $\hat{k}_{ij} := k_{ij}/\sqrt{k_{ii} k_{jj}}$, and then
averaged together to produce the {\em average kernel} used for learning.

\stefano{describe pairwise (tensor product) kernel, add references}
%
$$ k_{(i,j),(n,m)} := k_{in}k_{jm} + k_{im}k_{jn} $$

The single-protein and protein--protein average kernels were normalized and
preconditioned by up to $10^{-6}$, if necessary.

\subsection{Evaluation setting}

\begin{itemize}

    \item PF and PPI balanced fold construction (each test protein has
    at least one unknown interaction that we aim at predicting)

    \item \cite{park2012flaws}

    \item The results are available at \url{WRITEME}.

\end{itemize}

\paragraph{Protein function prediction}

\begin{itemize}

    \item Competitors: BLAST sequence transfer and nearest neighbor, as in
    CAFA2~\cite{jiang2016cafa2}; GoFDR~\cite{gong2016gofdr}; possibly other high-rank CAFA2 methods.

    \item Performance metrics: as in CAFA2, protein-centric ($F_\text{max}$ and $S_\text{min}$)
    and term-centric (AUC)~\cite{jiang2016cafa2}.

\end{itemize}

\paragraph{Protein--protein interaction prediction}

\begin{itemize}

    \item Competitors: ?

    \item Performance metrics: ?

\end{itemize}



\section{Discussion}

Make sure to mention that our method is tailored towards genome-wide
prediction, so CAFA2 is out of scope.

WRITEME



\section{Conclusion}

WRITEME



\section*{Acknowledgements}

WRITEME



\paragraph{Funding\textcolon}

WRITEME



\bibliographystyle{unsrt} % FIXME use bioinformatics.bst
\bibliography{document}

\end{document}
