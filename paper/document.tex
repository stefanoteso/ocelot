\documentclass{bioinfo}
\copyrightyear{2016}
\pubyear{2016}

\usepackage{hyperref}

\usepackage[usenames,dvipsnames]{xcolor}
\definecolor{azure}{rgb}{0.0, 0.5, 1.0}
\newcommand{\andrea}[1]{{\bf \textcolor{azure}{Andrea: #1}}}
\definecolor{alizarin}{rgb}{0.82, 0.1, 0.26}
\newcommand{\stefano}[1]{{\bf \textcolor{alizarin}{{Stefano: #1}}}}
\definecolor{antiquefuchsia}{rgb}{0.57, 0.36, 0.51}
\newcommand{\luca}[1]{{\bf \textcolor{antiquefuchsia}{Luca: #1}}}
\definecolor{applegreen}{rgb}{0.55, 0.71, 0.0}
\newcommand{\michleangelo}[1]{{\bf \textcolor{applegreen}{Michelangelo: #1}}}
\definecolor{munsell}{rgb}{0.0, 0.5, 0.69}
\newcommand{\claudio}[1]{{\bf \textcolor{munsell}{{Claudio: #1}}}}



\begin{document}
\firstpage{1}

\title[short Title]{Integrated Prediction of Protein Function and Interactions}
\author[Sample \textit{et~al}]{Claudio Sacc\`a\,$^{1}$, Stefano Teso\,$^{2}$, Michelangelo Diligenti\,$^{1*}$, and Andrea Passerini\,$^2$\footnote{to whom correspondence should be addressed}}
\address{
    $^{1}$Dipartimento di Ingegneria dell'Informazione e Scienze Matematiche, University of Siena, Siena, Italy\\
    $^{2}$Dipartimento di Ingegneria e Scienza dell'Informazione, University of Trento, Trento, Italy
}
\history{Received on XXXXX; revised on XXXXX; accepted on XXXXX}
\editor{Associate Editor: XXXXXXX}

\maketitle



\begin{abstract}
\section{Motivation:}
\section{Results:}
\section{Availability:}
\section{Contact:}
\end{abstract}



\section{Introduction}

\begin{itemize}

    \item Importance of genome-wide prediction of interactions and function

    \item Brief recap on protein function and GO hierarchy

    \item Brief recap on protein--protein interactions

    \item Relationships between function and interactions

    \item Statistical prediction methods for relational data with multiple
    features

    \item Structure of the genome-wide predictor (intro to SBR)

    \item Results highlights

    \item Structure of the paper

\end{itemize}



\begin{methods}

\section{Methods}

\subsection{Prediction Model}

\begin{itemize}

    \item High-level overview of joint function-interaction prediction at
    the genome scale

    \item We expand on the approach of Sacc\`a and colleagues~\cite{sacca2014ppisbr}

    \item Kernel-based predictions

    \item Constraints for function and interaction prediction

    \item Pairwise kernels

\end{itemize}

\subsection{Mathematical Formulation}

WRITEME

\end{methods}



\section{Results}

\subsection{Dataset construction}

In order to evaluate our proposed method, we built a comprehensive genome-wide
dataset. All data was retrieved in August 2014.

\paragraph{Annotations.}

The protein sequences and their GO annotations were taken from the
Saccharomyces Genome Database (SGD)~\cite{cherry2012sgd}. The GO hierarchy
definition was taken from the Gene Ontology Consortium website\footnote{
\url{http://geneontology.org/page/download-ontology}}. Only validated ORFs at
least 50 residues long were retained. The sequences were redundancy reduced
with CD-HIT~\cite{fu2012cdhit}, using a threshold of 80, producing a set of
4865 proteins. Following common practice~\cite{gong2016gofdr} \stefano{more
references}, we removed automatically assigned (IEA) GO annotations.

\stefano{add annotation statistics here}

Physical protein--protein interactions were taken from
BioGRID~\cite{chatr2015biogrid} (via SGD) and symmetrized, for a total of 34611
interactions. Due to the lack of shared repositories of known non-interacting
protein pairs, we resorted to a sampling procedure to produce high-quality
putative non-interactions, as follows.  \stefano{reference papers that justify
the procedure}.  First, we computed the complement of the positive PIN. In
order to minimize the chance of sampling subtracted the full set of functional
interactions (either physical or functional) in
STRING~\cite{franceschini2013string91,szklarczyk2014string10} (version 9.1)
from the complement. Finally, we sampled 34611 negative interactions from the
remaining candidate pairs. The full set of interaction annotations is taken to
be the union of the above positive and negative sets.

\paragraph{Kernels.}

In order to drive our learning procedure, we computed a number of kernels
between proteins and protein--protein pairs. We focused on protein features
that were shown to contribute useful information in several studies, including
genetic context, gene coexpression, known protein complexes, inferred
protein domains, and sequence profiles. We illustrate the computational
procedure in the following.

The genetic context of two proteins plays an important role in determining the
likelihood of them having physical interactions. We included this information
in the colocalization kernel, defined as follows:
%
$$ k_{ij} := \exp\left( - \gamma | x_i - x_j | \right) $$
%
Here the $x_i$ and $x_j$ are the position of the genes encoding proteins $p_i$
and $p_j$, taken to be the center of the two genes. Proteins encoded by
different chromosomes have zero similarity.

Gene expression levels in cell-cycle and environmental response were also shown
to provide invaluable information. We computed a correlation kernel over the
expression levels measured in two different experiments
(\cite{spellman1998comprehensive} and \cite{gasch2000genomic}).

We computed a diffusion kernel~\cite{kondor2002diffusion} over
the yeast protein complex catalogue of~\cite{pu2009up}. It determines the
correlation between proteins through a diffusion process over the protein
complex network.

We computed a set/sparse kernel \stefano{only keep the better one} over
inferred protein domains, derived using InterPro~\cite{mitchell2015interpro}.

We computed a profile kernel~\cite{kuang2005profile,hamp2013accelerating} over
all the proteins. The PSSMs were obtained with PSI-BLAST (default parameters,
two iterations) over NCBI's NR database. Please see \cite{kuang2005profile}
for additional details.

All kernels were normalized prior to use, by applying the transformation
$\hat{k}_{ij} := k_{ij}/\sqrt{k_{ii} k_{jj}}$.

\stefano{describe pairwise (tensor product) kernel, add references}
%
$$ k_{(i,j),(n,m)} := k_{in}k_{jm} + k_{im}k_{jn} $$

The source code of the data processing pipeline, as well as the resulting
dataset, are available at \url{WRITEME}.

\subsection{Evaluation setting}

\begin{itemize}

    \item Negative functions XXX

    \item Negative interactions sampled from the complement of the known
    positive interactions, minus the STRING database to avoid possible
    positives.

    \item PF and PPI balanced fold construction (each test protein has
    at least one unknown interaction that we aim at predicting)

    \item The results are available at \url{WRITEME}.

\end{itemize}

\paragraph{Protein function prediction}

\begin{itemize}

    \item Competitors: BLAST sequence transfer and nearest neighbor, as in
    CAFA2~\cite{jiang2016cafa2}; GoFDR~\cite{gong2016gofdr}; possibly other high-rank CAFA2 methods.

    \item Performance metrics: as in CAFA2, protein-centric ($F_\text{max}$ and $S_\text{min}$)
    and term-centric (AUC)~\cite{jiang2016cafa2}.

\end{itemize}

\paragraph{Protein--protein interaction prediction}

\begin{itemize}

    \item Competitors: ?

    \item Performance metrics: ?

\end{itemize}



\section{Discussion}

Make sure to mention that our method is tailored towards genome-wide
prediction, so CAFA2 is out of scope.

WRITEME



\section{Conclusion}

WRITEME



\section*{Acknowledgements}

WRITEME



\paragraph{Funding\textcolon}

WRITEME



\bibliographystyle{unsrt} % FIXME use bioinformatics.bst
\bibliography{document}

\end{document}
